\documentclass[UTF8]{ctexart}
\date{}
\title{\vspace{-4em} 推理系统常用结论证明 \vspace{-2em} }
\pagestyle{plain}
% \usepackage{ctex}
\usepackage{fancyhdr}
\usepackage[a4paper]{geometry}
\usepackage{multicol}
\usepackage{amsfonts}
\usepackage{amsmath}
\usepackage{listings}
\usepackage{graphicx}
\usepackage{amssymb}
\usepackage{algorithm}
\usepackage{algorithmicx}
\usepackage{algpseudocode}
\usepackage{float}
\usepackage{lipsum}
\usepackage{parskip}
\usepackage{indentfirst}
\usepackage{listings}
\usepackage{xcolor}
\usepackage{fbox}
% \usepackage{titlesec}

\CTEXsetup[format={\Large\bfseries}]{section}
\setlength{\parindent}{2em}
\setlength{\parskip}{0pt}

\begin{document}
    \maketitle

    \setlength{\abovedisplayskip}{0pt}
    \setlength{\belowdisplayskip}{0pt}
    
    \section{命题逻辑 \vspace{-3pt}}

    \subsection*{推理规则 $(\lnot^{+})$ \vspace{-3pt}}
    唯一一个用来给 $\vdash$ 右侧添加 $\lnot$ 符号的推理规则。

    \begin{equation*}
        (\lnot^{+}): \dfrac{\begin{matrix}\Sigma, A \vdash B \\ \Sigma, A \vdash \lnot B \end{matrix}}{\Sigma \vdash \lnot A}
    \end{equation*}

    \begin{align*}
        (1) && \Sigma, \lnot \lnot A & \vdash \Sigma & (\in) \\
        (2) && \lnot \lnot A, \lnot A & \vdash \lnot \lnot A & (\in) \\
        (3) && \lnot \lnot A, \lnot A & \vdash \lnot A & (\in) \\
        (4) && \lnot \lnot A & \vdash A & (\lnot^{-}) \\
        (5) && \Sigma, \lnot \lnot A & \vdash A & (+\ (4)) \\
        (6) && \Sigma, A & \vdash B & (\text{假设}) \\
        (7) && \Sigma, \lnot \lnot A & \vdash B & (\mathrm{Tr}\ (5)(6)) \\
        (8) && \Sigma, \lnot \lnot A & \vdash \lnot B & (\mathrm{ibid}) \\
        (9) && \Sigma & \vdash \lnot A & (\lnot^{-}\ (7)(8))
    \end{align*}

    % \subsection*{逆反推理规则 $\dfrac{A \vdash B}{\lnot B \vdash \lnot A}$}

    % \begin{align*}
    %     \lnot B, A & \vdash \lnot B & (\in) \\
    %     \lnot B, A & \vdash A & (\in) \\
    %     \lnot B, A & \vdash B & (+) \\
    %     \lnot B & \vdash \lnot A & (\lnot^{-})
    % \end{align*}
    
    \subsection*{$\lnot \lnot A \vdash A$}
    双重否定,书上 5.3.3 。类似还有 $A \vdash \lnot \lnot A$ ,用到 $(\lnot^{+})$ 。
    \begin{align*}
        \lnot \lnot A, \lnot A & \vdash \lnot \lnot A & (\in) \\
        \lnot \lnot A, \lnot A & \vdash \lnot A & (\in) \\
        \lnot \lnot A & \vdash A & (\lnot^{-})
    \end{align*}
    \subsection*{$\lnot A \vdash A \to B$}
    前件为假则蕴含式必为真,作业二第一题中涉及。
    \begin{align*}
        A, \lnot A, \lnot B & \vdash A & (\in) \\
        A, \lnot A, \lnot B & \vdash \lnot A & (\in) \\
        A, \lnot A & \vdash B & (\lnot^{-}) \\
        \lnot A & \vdash A \to B & (\to^{+})
    \end{align*}

    \subsection*{$A \to B \vdash (\lnot B) \to (\lnot A)$}
    逆否命题和原命题等价,作业二第 2 题用到。
    \begin{align*}
        A \to B, \lnot B, A & \vdash A \to B & (\in) \\
        A \to B, \lnot B, A & \vdash A & (\in) \\
        A \to B, \lnot B, A & \vdash B & (\lnot^{-}) \\
        A \to B, \lnot B, A & \vdash \lnot B & (\in) \\
        A \to B, \lnot B & \vdash \lnot A & (\lnot^{-}) \\
        A \to B & \vdash (\lnot B) \to (\lnot A) & (\to^{+})
    \end{align*}

    \subsection*{$A \vdash (\lnot A) \to B$}
    \begin{align*}
        A, \lnot A, \lnot B & \vdash A & (\in) \\
        A, \lnot A, \lnot B & \vdash \lnot A & (\in) \\
        A, \lnot A & \vdash B & (\lnot^{-}) \\
        A & \vdash (\lnot A) \to B & (\to^{+})
    \end{align*}

    \subsection*{$B \vdash A \to B$}
    \begin{align*}
        B, A & \vdash B & (\in) \\
        B & \vdash A \to B & (\to^{+})
    \end{align*}

    \subsection*{$A \lor B \vdash (\lnot A) \to B$}

    \begin{align*}
        A, \lnot A, \lnot B & \vdash A & (\in) \\
        A, \lnot A, \lnot B & \vdash \lnot A & (\in) \\
        A, \lnot A & \vdash B & (\lnot^{-}) \\
        A & \vdash (\lnot A) \to B & (\to^{+}) \\
        B, \lnot A & \vdash B & (\in) \\
        B & \vdash (\lnot A) \to B & (\to^{+}) \\
        A \lor B & \vdash (\lnot A) \to B & (\lor^{-})
    \end{align*}

    % \subsection{$\lnot A \to B \vdash A \lor B$}

    % \begin{align*}
    %     A & \vdash A & (\in) \\
    %     A & \vdash A \lor B & (\lor^{+}) \\
        
    % \end{align*}
    
    \newpage
    \subsection*{$\lnot(A \to B) \vdash A$}
    尝试推出 $\lnot (A \to B), \lnot A \vdash A$ ,可借助 $\lnot A \vdash A \to B$ 。
    \begin{align*}
        (1) && \lnot A & \vdash A \to B & (\lnot A \vdash A \to B) \\
        (2) && \lnot (A \to B) & \vdash (A\to B) \to A & (\lnot A \vdash A \to B) \\
        (3) && \lnot (A \to B), \lnot A & \vdash A \to B & (+\ (1)) \\
        (4) && \lnot (A \to B), \lnot A & \vdash (A \to B) \to A & (+\ (2)) \\
        (5) && \lnot (A \to B), \lnot A & \vdash A & (\to^{-}\ (3)(4)) \\
        (6) && \lnot (A \to B), \lnot A & \vdash \lnot A & (\in) \\
        (7) && \lnot (A \to B) & \vdash A & (\lnot^{-}\ (5)(6))
    \end{align*} \par
    完整版(上述步骤展开 $\lnot A \vdash A \to B$)
    \begin{align*}
        (1) && A, \lnot A, \lnot B & \vdash A & (\in) \\
        (2) && A, \lnot A, \lnot B & \vdash \lnot A & (\in) \\
        (3) && A, \lnot A & \vdash B & (\lnot^{-}) \\
        (4) && \lnot A & \vdash A \to B & (\to^{+}) \\
        (5) && (A \to B), \lnot (A \to B), \lnot A & \vdash (A \to B) & (\in) \\
        (6) && (A \to B), \lnot (A \to B), \lnot A & \vdash \lnot (A \to B) & (\in) \\
        (7) && (A \to B), \lnot (A \to B) & \vdash A & (\lnot^{-}) \\
        (8) && \lnot (A \to B) & \vdash (A \to B) \to A & (\to^{+}) \\
        (9) && \lnot (A \to B), \lnot A & \vdash A \to B & (+\ (4)) \\
        (10) && \lnot (A \to B), \lnot A & \vdash (A \to B) \to A & (+\ (8)) \\
        (11) && \lnot (A \to B), \lnot A & \vdash A & (\to^{-}\ (9)(10)) \\
        (12) && \lnot (A \to B), \lnot A & \vdash \lnot A & (\in) \\
        (13) && \lnot (A \to B) & \vdash A & (\lnot^{-}\ (11)(12))
    \end{align*}

    \subsection*{$\lnot (A \to B) \vdash \lnot B$}
    用 $(\lnot^{+})$ ,把 $B$ 提到左边,尝试构造右边是 $A \to B$ 和 $\lnot(A \to B)$ 。
    \begin{align*}
        A, B & \vdash B & (\in) \\
        B & \vdash A \to B & (\to^{+}) \\
        \lnot (A \to B), B & \vdash A \to B & (+) \\
        \lnot (A \to B), B & \vdash \lnot (A \to B) & (\in) \\
        \lnot (A \to B) & \vdash \lnot B & (\lnot^{+})
    \end{align*}

    \newpage
    
    \section{谓词逻辑}

    \subsection{$\lnot \forall x A(x) \to \exists x \lnot A(x)$}
    书上命题 5.3.3 第 1 条,给了例子。

    \begin{align*}
        \lnot A(z) & \vdash \lnot A(z) & (\in) \\
        \lnot A(z) & \vdash \exists x \lnot A(x) & (\exists^{+}) \\
        \lnot \exists x \lnot A(x), \lnot A(z) & \vdash \exists x \lnot A(x) & (+) \\
        \lnot \exists x \lnot A(x), \lnot A(z) & \vdash \lnot \exists x \lnot A(x) & (\in) \\
        \lnot \exists x \lnot A(x) & \vdash A(z) & (\lnot^{-}) \\
        \lnot \exists x \lnot A(x) & \vdash \forall x A(x) & (\forall^{+}) \\
        \lnot \forall x A(x), \lnot \exists x \lnot A(x) & \vdash \forall x A(x) & (+) \\
        \lnot \forall x A(x), \lnot \exists x \lnot A(x) & \vdash \lnot \forall x A(x) & (\in) \\
        \lnot \forall x A(x) & \vdash \exists x \lnot A(x) & (\lnot^{-})
    \end{align*}

    \subsection{$\lnot \exists x A(x) \to \forall x \lnot A(x)$}

    书上命题 5.3.3 第 2 条。
    \begin{align*}
        \lnot A(z) & \vdash \lnot A(z) & (\in) \\
        \lnot A(z) & \vdash \forall x \lnot A(x) & (\forall^{+}) \\
        \lnot \forall x \lnot A(x), \lnot A(z) & \vdash \forall x \lnot A(x) & (+) \\
        \lnot \forall x \lnot A(x), \lnot A(z) & \vdash \lnot \forall x \lnot A(x) & (\in) \\
        \lnot \forall x \lnot A(x) & \vdash A(z) & (\lnot^{-}) \\
        \lnot \forall x \lnot A(x) & \vdash \exists x A(x) & (\exists^{+}) \\
        \lnot \exists x A(x), \lnot \forall x \lnot A(x) & \vdash \exists x A(x) & (+) \\
        \lnot \exists x A(x), \lnot \forall x \lnot A(x) & \vdash \lnot \exists x A(x) & (\in) \\
        \lnot \exists x A(x) & \vdash \forall x \lnot A(x) & (\lnot^{-})
    \end{align*}

\end{document}