\documentclass[UTF8]{ctexart}
\author{}
\date{}
\title{程序理论-第一次作业}
\pagestyle{plain}
\usepackage{xeCJK}
\usepackage{fancyhdr}
\usepackage[a4paper,scale=0.75]{geometry}
\usepackage{multicol}
\usepackage{amsfonts}
\usepackage{amsmath}
\usepackage{listings}
\usepackage{graphicx}
\usepackage{amssymb}
\usepackage{algorithm}
\usepackage{algorithmicx}
\usepackage{algpseudocode}
\usepackage{float}
\usepackage{lipsum}
\usepackage{parskip}
\usepackage{indentfirst}
\usepackage{listings}
\usepackage{xcolor}
\usepackage{fbox}
\usepackage{titling}
\usepackage{titlesec}
\usepackage{framed}
\usepackage{stmaryrd}

\CTEXsetup[format={\Large\bfseries}]{section}
\setlength{\parskip}{0pt}
\setlength{\droptitle}{-4em}

\begin{document}
    \maketitle

    \setlength{\abovedisplayskip}{0pt}
    \setlength{\belowdisplayskip}{0pt}

    \section*{Notes}

    \subsection*{IMP 定义}
    
    共 5 种语句,其中 $var$ 为变量,$aexp$ 为算术表达式,$bexp$ 为布尔表达式:\vspace{-1em}

    \begin{align*}
        com &= \mathbf{skip}    & (\mathsf{skip})   \\
            &\mid var := aexp   & (\mathsf{assign}) \\
            &\mid com; com      & (\mathsf{seq})    \\
            &\mid \mathbf{if}\ bexp\ \mathbf{then}\ com\ \mathbf{else}\ com  & (\mathsf{if}) \\
            &\mid \mathbf{while}\ bexp\ \mathbf{do}\ com & (\mathsf{while})
    \end{align*}

    \subsection*{程序状态}

    变量到值的映射,例如 $s=\llparenthesis a:=2, b:=3, c:=0 \rrparenthesis$ ;用 ${\llbracket e \rrbracket}_s$ 表示表达式 $e$ 在状态 $s$ 的值。

    在 \LaTeX 中引入 \texttt{stmaryrd} 包,并在公式环境中使用 \texttt{\textbackslash llbracket \textbackslash rrbracket} 打出 $\llbracket\ \rrbracket$,使用 \texttt{\textbackslash llparenthesis \textbackslash rrparenthesis} 打出 $\llparenthesis\ \rrparenthesis$。

    \subsection*{大步操作语义 (Big-step operational semantics)}

    程序 $c$ 将状态 $s$ 转移到状态 $t$:$$(c, s) \Rightarrow t$$

    推理规则:

    \begin{itemize}
        \setlength{\itemsep}{1em}
        \setlength{\parsep}{1em}
        % \setlength{\topsep}{1em}
        \item $\dfrac{}{(\mathbf{skip}, s) \Rightarrow s}\ \mathsf{skip}$, $\dfrac{(c_1, s) \Rightarrow s'\quad (c_2,s') \Rightarrow s''}{(c_1;c_2,s) \Rightarrow s''}\ \mathsf{seq}$;
        \item $\dfrac{}{(v:=e, s) \Rightarrow s(v:= {\llbracket e \rrbracket}_s)}\ \mathsf{assign}$,其中 $s(v:= {\llbracket e \rrbracket}_s)$ 表示用 $v := {\llbracket e \rrbracket}_s$ 更新 $s$;
        \item $\dfrac{{\llbracket b \rrbracket}_s = \mathsf{true}\quad (c_1,s)\Rightarrow t}{(\mathbf{if}\ b\ \mathbf{then}\ c_1\ \mathbf{else}\ c_2, s) \Rightarrow t}\ \mathsf{ifTrue}$,$\dfrac{{\llbracket b \rrbracket}_s = \mathsf{false}\quad (c_2,s)\Rightarrow t}{(\mathbf{if}\ b\ \mathbf{then}\ c_1\ \mathbf{else}\ c_2, s) \Rightarrow t}\ \mathsf{ifFalse}$;
        \item $\dfrac{{\llbracket b \rrbracket}_s=\mathsf{false}}{(\mathbf{while}\ b\ \mathbf{do}\ c, s)\Rightarrow s}\ \mathsf{whileFalse}$
        \item $\dfrac{{\llbracket b \rrbracket}_s=\mathsf{true}\quad (c,s)\Rightarrow s'\quad (\mathbf{while}\ b\ \mathbf{do}\ c, s')\Rightarrow s''}{(\mathbf{while}\ b\ \mathbf{do}\ c, s)\Rightarrow s''}\ \mathsf{whileTrue}$
    \end{itemize}

    \subsection*{小步操作语义 (Small-step operational semantics)}

    程序 $c$ 执行一步,剩余部分为 $c'$,从状态 $s$ 转移到状态 $s'$:$$(c, s) \to (c', s')$$

    如果 $c$ 是 $\mathbf{skip}$ 表示程序终止了。

    推理规则:

    \begin{itemize}
        \setlength{\itemsep}{1em}
        \setlength{\parsep}{1em}
        % \setlength{\topsep}{1em}
        \item $\dfrac{}{(v:=e,s) \to (\mathbf{skip}, s(v:= {\llbracket e \rrbracket}_s))}\ \mathsf{assign}$
        \item $\dfrac{(c_1, s) \to (c_1', s')}{(c_1; c_2, s) \to (c_1'; c_2, s')}\ \mathsf{seq1}$,$\dfrac{}{(\mathbf{skip}; c_2, s) \to (c_2, s)}\ \mathsf{seq2}$
        \item $\dfrac{{\llbracket b \rrbracket}_s=\mathsf{true}}{(\mathbf{if}\ b\ \mathbf{then}\ c_1\ \mathbf{else}\ c_2, s) \to (c_1, s)}\ \mathsf{ifTrue}$,$\dfrac{{\llbracket b \rrbracket}_s=\mathsf{false}}{(\mathbf{if}\ b\ \mathbf{then}\ c_1\ \mathbf{else}\ c_2, s) \to (c_2, s)}\ \mathsf{ifFalse}$
        \item $\dfrac{}{(\mathbf{while}\ b\ \mathbf{do}\ c, s) \to (\mathbf{if}\ b\ \mathbf{then}\ (c; \mathbf{while}\ b\ \mathbf{do}\ c)\ \mathbf{else}\ \mathbf{skip}, s)}\ \mathsf{while}$
    \end{itemize}
    \setlength{\parskip}{1em}
    符号 $\to^{\ast}$ 表示 $\to$ 的自反传递闭包:$(c, s) \to^{\ast} (c', s')$ 表示 $(c, s)$ 执行零步或多步到达 $(c', s')$,$(c, s) \to^{\ast} (\mathbf{skip}, s')$ 表示从 $(c, s)$ 执行最终停止到状态 $s'$ 。
    \setlength{\parskip}{0em}

    自反传递闭包 $\to^{\ast}$ 的推理规则(归纳推理):

    \begin{itemize}
        \setlength{\itemsep}{1em}
        \item 自反性:$\dfrac{}{(c, s) \to^{\ast} (c, s)}$
        \item 传递性:$\dfrac{(c_1, s_1) \to (c_3, s_3)\quad (c_3, s_3) \to^{\ast} (c_2, s_2)}{(c_1, s_1) \to^{\ast} (c_2, s_2)}$
    \end{itemize}

    \subsection*{大步小步语义的等价性}

    课堂上讲了从大步语义 $(c, s) \Rightarrow t$ 推出小步语义 $(c, s) \to^{\ast} (\mathbf{skip}, t)$ 的证明过程:

    \begin{framed}
        证明由 $(c, s) \Rightarrow t$ 得出 $(c, s) \to^{\ast} (\mathbf{skip}, t)$ 。

        首先需要证明小步语义的一个引理:$(c_1, s_1) \to^{\ast} (c_2, s_2)$ 推出 $(c_1; c', s_1) \to^{\ast} (c_2; c', s_2)$ 。

        \begin{itemize}
            \item 基础:由自反性 $(c, s) \to^{\ast} (c, s)$ 有 $(c; c', s) \to^{\ast} (c; c', s)$。
            \item 归纳:假定有 $(c_1, s_1) \to (c_3, s_3)$ 和 $(c_3, s_3) \to^{\ast} (c_2, s_2)$,需要证 $(c_1; c', s_1) \to^{\ast} (c_2; c', s2)$。
            \begin{itemize}
                \item 由小步语义推理规则 $\mathrm{seq1}$ 有 $(c_1; c', s_1) \to (c_3; c', s_3)$;
                \item 由归纳假设,根据 $(c_3, s_3) \to^{\ast} (c_2, s_2)$ 有 $(c_3; c', s_3) \to^{\ast} (c_2; c', s_2)$;
                \item 由 $\to^{\ast}$ 的传递性,得 $(c_1; c', s_1) \to^{\ast} (c_2; c', s_2)$。
            \end{itemize}
        \end{itemize}

        按大步语义推理的步数归纳 $(c, s) \Rightarrow t$,根据最后使用的大步语义推理规则划分情况:

        \begin{itemize}
            \item $\mathsf{skip}$:当 $c=\mathbf{skip}$ 且 $s=t$,只需证 $(\mathbf{skip}, s) \to^{\ast} (\mathbf{skip}, s)$,由 $\to^{\ast}$ 自反性可得。
            \item $\mathsf{assign}$:当 $c=(v:=e)$ 且 $t=s(v:={\llbracket e \rrbracket}_s)$,只需证 $(v:=e,s) \to^{\ast} (\mathbf{skip}, s(v:={\llbracket e \rrbracket}_s))$,由小步语义的 $\mathsf{assign}$ 推理规则可得。
            \item $\mathsf{seq}$:当 $c=c_1; c_2$ 并且有状态 $s'$ 使得 $(c_1, s)\Rightarrow s'$ 且 $(c_2, s') \Rightarrow t$。由归纳假设,有 $(c_1, s) \to^{\ast} (\mathbf{skip}, s')$ 和 $(c_2, s') \to^{\ast} (\mathbf{skip}, t)$。根据引理,有 $(c_1; c_2, s) \to^{\ast} (\mathbf{skip}; c_2, s')$,结合小步语义 $\mathsf{seq2}$ 推理规则,得 $(c_1; c_2, s) \to^{\ast} (\mathbf{skip}, t)$。
            \item $\mathsf{ifTrue}$:当 $c=\mathbf{if}\ b\ \mathbf{then}\ c_1\ \mathbf{else}\ c_2$,${\llbracket b \rrbracket}_s=\mathsf{true}$ 且 $(c_1, s) \Rightarrow t$,由归纳假设有 $(c_1, s) \to^{\ast} (\mathbf{skip}, t)$,结合小步语义 $\mathsf{ifTrue}$ 推理规则可得 $(c, s) \to^{\ast} (\mathbf{skip}, t)$。$\mathsf{ifFalse}$ 情况类似可证。
            \item $\mathsf{whileFalse}$:当 $c=\mathbf{while}\ b\ \mathbf{do}\ c'$,${\llbracket b \rrbracket}_s=\mathsf{false}$ 且 $s=t$,由小步语义的 $\mathsf{while}$ 和 $\mathsf{ifFalse}$ 推理规则,可得 $(c, s) \to^{\ast} (\mathbf{skip}, t)$。
            \item $\mathsf{whileTrue}$:当 $c=\mathbf{while}\ b\ \mathbf{do}\ c'$,${\llbracket b \rrbracket}_s=\mathsf{true}$, $(c', s) \Rightarrow s'$ 且 $(c, s') \Rightarrow t$。根据归纳假设,有 $(c', s) \to^{\ast} (\mathbf{skip}, s')$ 和 $(c, s') \to^{\ast} (\mathbf{skip}, t)$。根据引理有 $(c';c, s) \to^{\ast} (\mathbf{skip};c, s')$,结合小步语义的 $\mathsf{while},\mathsf{ifTrue},\mathsf{seq2}$ 推理规则得 $(c, s) \to^{\ast} (\mathbf{skip}, t)$。
        \end{itemize}
        综上,对于任何的 $(c, s) \Rightarrow t$ 都可以推出 $(c, s) \to^{\ast} (\mathbf{skip}, t)$。
    \end{framed}

    \newpage
    
    \section*{Written Part}

    \subsection*{Problem 1}

    \begin{framed}
    Carefully write down (as in the lecture slides)  the other direction in the equivalence between big-step and small-step semantics.

    从小步语义 $(c, s) \to^{\ast} (\mathbf{skip}, t)$ 推出大步语义 $(c, s) \Rightarrow t$。
    \end{framed}

    解:与课上讲的大步语义证小步语义类似的归纳。如果有 $(c_1, s_1) \to (c_2, s_2)$,并且 $(c_2, s_2) \to^{\ast} (\mathbf{skip}, t)$ ,根据归纳假设有 $(c_2, s_2) \Rightarrow t$;由 $\to^{\ast}$ 的传递推理规则可以得到 $(c_1, s_1) \to^{\ast} (\mathbf{skip}, t)$。接下来只需要根据 $(c_1, s_1) \to (c_2, s_2)$ 所采用的小步语义推理规则分类讨论,并转到对应的大步推理规则从 $(c_2, s_2) \Rightarrow t$ 推出 $(c_1, s_1) \Rightarrow t$ 即可归纳得到对 $(c_1, s_1) \to^{\ast} (\mathbf{skip}, t)$ 也能推出 $(c_1, s_1) \Rightarrow t$。
    
    基础情况:当 $c=\mathbf{skip}, s=t$ 时,由 $\to^{\ast}$ 的自反性,$(c, s)\to^{\ast}(\mathbf{skip}, t)$,由大步语义的 $\mathsf{skip}$ 推理规则,$(\mathbf{skip}, s) \Rightarrow t$,此时有 $(c, s) \Rightarrow t$ 成立。

    归纳推理:按最后一步采用的小步语义推理规则分类

    \begin{itemize}
        \item $\mathsf{assign}$:当 $c=(v:=e)$ 且 $t=s(v:={\llbracket e \rrbracket}_s)$,有 $(v:=e; s) \to (\mathbf{skip}, s(v:={\llbracket e \rrbracket}_s))$。只需证 $(v:=e,s)\Rightarrow s(v:={\llbracket e \rrbracket}_s)$,由大步语义的 $\mathsf{assign}$ 推理规则可得。
        \item $\mathsf{seq2}$:当 $c=\mathbf{skip}; c_2$ 时,有 $(\mathbf{skip}; c_2, s) \to (c_2, s)$。根据归纳假设,由 $(c_2, s) \to^{\ast} (\mathbf{skip}, t)$ 得 $(c_2, s) \Rightarrow t$;根据 $\to^{\ast}$ 的传递性,有 $(\mathbf{skip}; c_2, s) \to^{\ast} (\mathbf{skip}, t)$。只需证 $(\mathbf{skip}; c_2, s) \Rightarrow t$,可根据大步语义的 $\mathsf{skip}, \mathsf{seq}$ 推理规则,由 $(\mathbf{skip}, s) \Rightarrow s$ 和 $(c_2, s) \Rightarrow t$ 得证。
        \item $\mathsf{seq1}$:当 $c=c_1; c_2$ 且 $(c_1, s) \to (c_1', s')$,有 $(c_1; c_2, s) \to (c_1'; c_2, s')$。根据归纳假设,如果 $(c_1'; c_2, s') \to^{\ast} (\mathbf{skip}, t)$ 则 $(c_1'; c_2, s') \Rightarrow t$,只需证 $(c_1; c_2, s) \Rightarrow t$。由大步语义的 $\mathsf{seq}$ 规则,存在 $s''$ 使得 $(c_1', s') \Rightarrow s''$ 且 $(c_2, s'') \Rightarrow t$。根据课上证过的从大步语义到小步语义的等价性,由 $(c_1', s') \Rightarrow s''$ 可以得 $(c_1', s') \to^{\ast} (\mathbf{skip}, s'')$,再根据 $(c_1, s) \to (c_1', s')$ 和传递性得 $(c_1, s) \to^{\ast} (\mathbf{skip}, s'')$,由归纳假设得到 $(c_1, s) \Rightarrow s''$。再由 $(c_2, s'') \Rightarrow t$ 和大步语义的 $\mathsf{seq}$ 推理规则得到 $(c_1; c_2, s) \Rightarrow t$。
        \item $\mathsf{ifTrue}$:当 $c=\mathbf{if}\ b\ \mathbf{then}\ c_1\ \mathbf{else}\ c_2$ 且 ${\llbracket b \rrbracket}_s = \mathsf{true}$,有 $(c, s) \to (c_1, s)$。如果 $(c_1, s) \to^{\ast} (\mathbf{skip}, t)$ 则由传递性有 $(c, s) \to^{\ast} (\mathbf{skip}, t)$,再由归纳假设有 $(c_1, s) \Rightarrow t$。根据大步语义的 $\mathsf{ifTrue}$ 推理规则,可得 $(c, s) \Rightarrow t$。$\mathsf{ifFalse}$ 与 $\mathsf{ifTrue}$ 类似。
        \item $\mathsf{while}$:当 $c=\mathbf{while}\ b\ \mathbf{do}\ c'$ 有 $(c, s) \to (\mathbf{if}\ b\ \mathbf{then}\ (c'; \mathbf{while}\ b\ \mathbf{do}\ c')\ \mathbf{else}\ \mathbf{skip}, s)$。\\
        当 ${\llbracket b \rrbracket}_s = \mathsf{false}$ 时,根据大步语义的 $\mathsf{whileFalse}$ 推理规则,有 $(c, s) \Rightarrow s$。\\
        当 ${\llbracket b \rrbracket}_s = \mathsf{true}$ 时,当 $(c'; \mathbf{while}\ b\ \mathbf{do}\ c', s) \to^{\ast} (\mathbf{skip}, t)$ 时,由小步语义的 $\mathsf{ifTrue}$ 规则有 $(\mathbf{if}\ b\ \mathbf{then}\ (c'; \mathbf{while}\ b\ \mathbf{do}\ c')\ \mathbf{else}\ \mathbf{skip}, s) \to^{\ast} (\mathbf{skip}, t)$,再由传递性有 $(c, s) \to^{\ast} (\mathbf{skip}, t)$。此时只需证 $(c, s) \Rightarrow t$。根据归纳假设,$(c'; \mathbf{while}\ b\ \mathbf{do}\ c', s) \Rightarrow t$,由大步语义的 $\mathsf{seq}$ 推理规则,存在 $s'$ 使得 $(c', s) \Rightarrow s'$ 并且 $(\mathbf{while}\ b\ \mathbf{do}\ c', s') \Rightarrow t$;再由大步语义的 $\mathsf{whileTrue}$ 推理规则,有 $(\mathbf{while}\ b\ \mathbf{do}\ c', s') \Rightarrow t$,即 $(c, s) \Rightarrow t$ 得证。
    \end{itemize}

    \newpage
    
    \subsection*{Problem 2}

    \begin{framed}
        Define (by induction on the structure of the program) a function $assigned$ from programs to a set of variables that may be assigned to in the program. Prove (by induction on big-step semantics) that if $(c, s) \Rightarrow t$ and $x$ are not in the assigned set, then the value of $x$ in $s$ and $t$ are the same.
    \end{framed}

    解:要定义的函数 $assigned$ 参数为一个 IMP 程序,返回值为一个集合。形式化的定义如下:

    $$
    assigned(com) = \begin{cases}
        \varnothing & com=\mathbf{skip} \\
        \left\{var\right\} & com=(var := aexp) \\
        assigned({com}_1) \cup assigned({com}_2) & com=({com}_1; {com}_2) \\
        assigned({com}_1) \cup assigned({com}_2) & com=(\mathbf{if}\ bexp\ \mathbf{then}\ {com}_1\ \mathbf{else}\ {com}_2) \\
        assigned(com) & com=(\mathbf{while}\ bexp\ \mathbf{do}\ com)
    \end{cases}
    $$

    要证明的结论:如果 $(c, s) \Rightarrow t$ 并且 $x \notin assigned(c)$ 则 ${\llbracket x \rrbracket}_s = {\llbracket x \rrbracket}_t$。

    归纳证明:
    \begin{itemize}
        \item $c=\mathbf{skip}$ 则一定有 $t=s$ 和 $assigned(c) = \varnothing$,显然 ${\llbracket x \rrbracket}_s = {\llbracket x \rrbracket}_t$ 成立。
        \item $c=(v:=e)$,此时 $t=s(v:={\llbracket e \rrbracket}_s)$ 且 $assigned(c)=\left\{v\right\}$。若 $x \notin assigned(c)$ 则 $x \ne v$,${\llbracket x \rrbracket}_s = {\llbracket x \rrbracket}_{s(v:={\llbracket e \rrbracket}_s)} = {\llbracket x \rrbracket}_t$,结论成立。
        \item $c=(c_1; c_2)$,设 $(c_1, s) \Rightarrow s', (c_2, s') \Rightarrow t$。根据归纳假设:\begin{itemize}
            \item 当 $x \notin assigned(c_1)$ 时有 ${\llbracket x \rrbracket}_s = {\llbracket x \rrbracket}_{s'}$
            \item 当 $x \notin assigned(c_2)$ 时有 ${\llbracket x \rrbracket}_{s'} = {\llbracket x \rrbracket}_t$
        \end{itemize}
        因为 $assigned(c)=assigned(c_1) \cup assigned(c_2)$,$x \notin assigned(c)$ 意味着 $x \notin assigned(c_1)$ 且 $x \notin assigned(c_2)$,所以 ${\llbracket x \rrbracket}_s = {\llbracket x \rrbracket}_{s'} = {\llbracket x \rrbracket}_t$。
        \item $c=(\mathbf{if}\ b\ \mathbf{then}\ c_1\ \mathbf{else}\ c_2)$,此时 $assigned(c)=assigned(c_1) \cup assigned(c_2)$,$x \notin assigned(c)$ 意味着 $x \notin assigned(c_1)$ 且 $x \notin assigned(c_2)$。\\
        对 $b$ 分类讨论:\begin{itemize}
            \item ${\llbracket b \rrbracket}_s=\mathsf{true}$,由 $\mathsf{ifTrue}$ 推理规则,设 $(c_1, s) \Rightarrow t$ 则 $(c, s) \Rightarrow t$。\\
            根据归纳假设,由 $x \notin assigned(c_1)$ 可推出 ${\llbracket x \rrbracket}_s = {\llbracket x \rrbracket}_t$。
            \item ${\llbracket b \rrbracket}_s=\mathsf{false}$,由 $\mathsf{ifFalse}$ 推理规则,设 $(c_2, s) \Rightarrow t$ 则 $(c, s) \Rightarrow t$。\\
            根据归纳假设,由 $x \notin assigned(c_2)$ 可推出 ${\llbracket x \rrbracket}_s = {\llbracket x \rrbracket}_t$。
        \end{itemize}
        无论 ${\llbracket b \rrbracket}_s$ 取何值,总有 ${\llbracket x \rrbracket}_s = {\llbracket x \rrbracket}_t$。
        \item $c=(\mathbf{while}\ b\ \mathbf{do}\ c')$,此时 $assigned(c)=assigned(c')$。对 $b$ 分类讨论:\begin{itemize}
            \item ${\llbracket b \rrbracket}_s=\mathsf{false}$,由 $\mathsf{whileFalse}$ 规则 $t=s$,显然 ${\llbracket x \rrbracket}_s = {\llbracket x \rrbracket}_t$ 成立。
            \item ${\llbracket b \rrbracket}_s=\mathsf{true}$,由 $\mathsf{whileTrue}$ 规则设 $(c', s) \Rightarrow s', (\mathbf{while}\ b\ \mathbf{do}\ c', s') \Rightarrow t$。根据归纳假设,对 $(c', s) \Rightarrow s'$ 当 $x \notin assigned(c')$ 时 ${\llbracket x \rrbracket}_s = {\llbracket x \rrbracket}_{s'}$;对 $(\mathbf{while}\ b\ \mathbf{do}\ c', s') \Rightarrow t$ 当 $x \notin assigned(c)$ 时 ${\llbracket x \rrbracket}_{s'} = {\llbracket x \rrbracket}_t$。结合两者有 ${\llbracket x \rrbracket}_s = {\llbracket x \rrbracket}_t$ 成立。
        \end{itemize}
        无论 ${\llbracket b \rrbracket}_s$ 取何值,总有 ${\llbracket x \rrbracket}_s = {\llbracket x \rrbracket}_t$。
    \end{itemize}

    \subsection*{Problem 3}

    \begin{framed}
        Define a small-step semantics for the evaluation of arithmetic expressions, specifying a left-to-right evaluation order. The syntax for arithmetic expressions is given by:

        $$
            expr = N\ int \mid V\ var \mid Plus\ expr\ expr
        $$

        where $N$ denotes constants, $V$ denotes variables, and $Plus$ denotes addition. For example, it should be possible to derive from the semantics the following:

        $$
            (Plus\ (Plus\ (N\ 3)\ (V\ x))\ (V\ y), \llparenthesis x:=5, y:=2 \rrparenthesis) \to^{\ast} (N\ 10, \llparenthesis x:=5, y:=2 \rrparenthesis)
        $$
    \end{framed}

    表达式求值过程的小步语义:$(e, s) \to (e', s)$ (显然状态 $s$ 不会被改变) \begin{itemize}
        \item $e=N\ n$,小步执行终止。更进一步地,任何 $e$ 都存在 $n$ 使得 $(e, s) \to^{\ast} (N\ n, s)$。
        \item $e=V\ v$,有 $(V\ v, s) \to (N\ {\llbracket v \rrbracket}_s, s)$。
        \item $e=Plus\ e_1\ e_2$,分三种情况 \begin{itemize}
            \item 两个操作数均已确定,即 $e_1=N\ n_1, e_2=N\ n_2$,则 $(e, s) \to (N\ (n_1+n_2), s)$
            \item 第一个操作数已确定,则将第二个操作数运算一步,即 $(e, s) \to (Plus\ e_1, e_2', s)$
            \item 两个操作数均未确定,则将第一个操作数运算一步,即 $(e, s) \to (Plus\ e_1', e_2, s)$
        \end{itemize}
    \end{itemize}

    用形式化的小步语义推理规则描述:\begin{itemize}
        \setlength{\itemsep}{1em}
        \setlength{\parsep}{1em}
        \item $(N\ n, s)$ 为终止状态,无推理规则
        \item $\dfrac{}{(V\ v, s) \to (N\ {\llbracket v \rrbracket}_s, s)}\ \mathsf{var}$
        \item $\dfrac{}{(Plus\ (N\ n_1)\ (N\ n_2), s) \to (N\ n_1+n_2, s)}\ \mathsf{plus2}$
        \item $\dfrac{(e_2, s) \to (e_2', s)}{(Plus\ (N\ n_1)\ e_2, s) \to (Plus\ (N\ n_1)\ e_2', s)}\ \mathsf{plus1}$
        \item $\dfrac{(e_1, s) \to (e_1', s)}{(Plus\ e_1\ e_2, s) \to (Plus\ e_1'\ e_2, s)}\ \mathsf{plus0}$
    \end{itemize}

    \setlength{\parskip}{1em}
    示例表达式的小步推理过程:
    \setlength{\parskip}{0em}

    \begin{align*}
       & (Plus\ (Plus\ (N\ 3)\ (V\ x))\ (V\ y), \llparenthesis x:=5, y:=2 \rrparenthesis) \\
       \to\ & (Plus\ (Plus\ (N\ 3)\ (N\ 5))\ (V\ y), \llparenthesis x:=5, y:=2 \rrparenthesis) & (\mathsf{plus0}, \mathsf{plus1}) \\
       \to\ & (Plus\ (N\ 8)\ (V\ y), \llparenthesis x:=5, y:=2 \rrparenthesis) & (\mathsf{plus0}, \mathsf{plus2}) \\
       \to\ & (Plus\ (N\ 8)\ (N\ 2), \llparenthesis x:=5, y:=2 \rrparenthesis) & (\mathsf{plus1}, \mathsf{var}) \\
       \to\ & (N\ 10, \llparenthesis x:=5, y:=2 \rrparenthesis) & (\mathsf{plus2})
    \end{align*}

\end{document}